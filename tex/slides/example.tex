% Start a slide
\begin{frame}[fragile]{Title of slide} % [fragile] needed when the content contains juliaverbatim.

Example text within slide.\footcite{Kochenderfer2019}

% Sample equation
\begin{equation*}
\underbrace{V(s)}_{\text{state value}} \!\!\! \leftarrow \;
\max\limits_a \overbrace{\biggl(
\underbrace{R(s,a)}_{\text{reward}} \; + \;
\underbrace{\gamma V(s^\prime)}_{\mathclap{\substack{\text{discounted value} \\ \text{of next state}}}}
\biggr)}^{\mathclap{\text{estimate of optimal discounted value}}}
\end{equation*}

% Julia code block
\begin{algorithm}
\begin{juliaverbatim}
# Julia function
V(s) = max(a->R(s,a) + γ*V(s′), A)
\end{juliaverbatim}
\end{algorithm}

\end{frame}



% Another slide (note, [fragile] is neede when using juliaconsole)
\begin{frame}[fragile]{Julia console}

Example Julia code executing at LaTeX compilation time. % \LaTeX{}

\begin{juliaconsole}
using LinearAlgebra
A = Matrix{Int}(I, 3, 3)
\end{juliaconsole}

\end{frame}



% Another slide (note, [fragile] is needed when using jlcode)
\begin{frame}[fragile]{Plotting}

Plot using \jlv{PGFPlots.jl}\footnote{\url{https://github.com/JuliaTeX/PGFPlots.jl}} directly in the TeX file.
\begin{jlcode}
    using Distributions

    function format_sigma(s)
        try
            return Int(s)
        catch e
            return s
        end
    end

    p = let
        p = GroupPlot(3, 1, groupStyle="horizontal sep=1cm, y descriptions at=edge left")
        for (i, (mu, sigma)) in enumerate([([0,0],[1 0; 0 1]),
                                           ([0,0],[1 0.75; 0.75 1]),
                                           ([0,0],[3 0; 0 3])])
            push!(p, Axis(
                Plots.Image((x,y)->pdf(MvNormal(mu,Float64.(sigma)), [x,y]), (-5,5), (-5,5),
                            zmin=0, zmax=0.05, xbins=151, ybins=151, colormap=pasteljet, colorbar=(i==4),
                            colorbarStyle="width=2mm"
                    ), height="4.5cm", width="4.5cm",
                    title="{\\footnotesize\\shortstack{\$\\vec \\mu=[$(mu[1]),$(mu[2])]\$\\\\\$\\mat \\Sigma=\\begin{bmatrix}$(format_sigma(sigma[1,1])) & $(format_sigma(sigma[1,2]))\\\\$(format_sigma(sigma[2,1])) & $(format_sigma(sigma[2,2]))\\end{bmatrix}\$}}",
                    style="colorbar style={scaled y ticks=false,yticklabel style={/pgf/number format/.cd, fixed, fixed zerofill, precision=2, /tikz/.cd}}"))
        end
        p
    end
    plot(p)
\end{jlcode}
\begin{figure}
    \begin{center}
        \plot{fig/gaussian-bivariate-example-stack}
    \end{center}
\end{figure}

\end{frame}



% Another slide
\begin{frame}{Example: bullet points}
    \begin{itemize}
        \item Bullet point
        \begin{itemize}
            \item Sub-bullet point
        \end{itemize}
        \item Bullet point
        \item Bullet point
    \end{itemize}
\end{frame}