\begin{frame}[fragile]{Belief Vector}

\begin{itemize}
    \item In the finite state case, we can represent beliefs using a categorical distribution\footnote{A probability mass is assigned to each discrete state.}
    \begin{itemize}
        \item Represented as a \textit{belief vector} $\vec{b}$ of length $\card{\mathcal{S}}$, therefore $\mathcal{B} \subset \R^{\card{\mathcal{S}}}$
        \item Sometimes $\mathcal{B}$ is referred to as a \textit{probability simplex} or \textit{belief simplex}\footnote{Simplex being the generalization of a triangle to arbitrary dimensions.}
    \end{itemize}
\end{itemize}

% \vspace{5mm}

\begin{itemize}
    \item The belief vector $\vec{b}$ must be strictly non-negative and sum to one:
    \begin{equation*}
        b(s) \ge 0 \text{ for all } s \in \mathcal{S} \qquad \sum_s b(s) = 1
    \end{equation*}
    \item In vector notation:
    \begin{equation*}
        \vec{b} \ge \vec{0} \qquad \vec{1}^\top \vec{b} = 1
    \end{equation*}
    \item In Julia syntax:
\begin{center}
\jlv{all(𝐛 .≥ 0) && sum(𝐛) ≈ 1}
\end{center}
\end{itemize}

\end{frame}